L'algoritmo di simulazione FIRE può essere concettualmente suddiviso in diversi
moduli chiave, ciascuno responsabile di un aspetto specifico del processo di
simulazione.
Questi moduli interagiscono tra loro per creare un modello completo
della performance del portafoglio.

    \subsection{Input e Configurazione dei
    Dati}\label{subsec:input-e-configurazione-dei-dati}

Questo modulo è responsabile della raccolta e della strutturazione dei dati di
input richiesti per la simulazione.
Questo include:
    \begin{itemize}
        \item \textbf{Valore Iniziale del Portafoglio:} Il valore iniziale del
        portafoglio di investimenti.
        \item \textbf{Riserve di Liquidità:} L'importo iniziale di liquidità
        detenuto al di fuori del portafoglio di investimenti.
        \item \textbf{Asset Allocation:} L'allocazione del portafoglio tra
        diverse classi di attività (ad esempio, azioni, obbligazioni, immobili).
        Questo influenza il rendimento atteso e la volatilità del portafoglio.
        \item \textbf{Strategia di Prelievo:} L'importo di prelievo annuale
        pianificato, potenzialmente adeguato per l'inflazione.
        Questo può essere
        un importo fisso, una percentuale del portafoglio o una strategia più
        complessa.
        \item \textbf{Modello di Inflazione:} Specifica di come vengono generati
        i tassi di inflazione (vedere la Sezione 3.3).
        \item \textbf{Parametri di Simulazione:} Parametri come la durata della
        simulazione (numero di anni), il numero di simulazioni da eseguire e il
        tasso di interesse sulla liquidità.
        \item \textbf{Tassi di Spesa:} Tassi di imposta, percentuali di imposta
        di bollo e percentuali di carico applicate ai valori degli investimenti
        o alle transazioni.
        \item \textbf{Impostazioni Intervallo di Pensionamento:} Specifica gli
        anni fino al pensionamento completo e il numero di anni nel
        pensionamento completo.
        \item \textbf{Dati Storici di Mercato:} Dati di serie temporali che
        rappresentano i rendimenti storici del mercato per le classi di attività
        scelte.
        \item \textbf{Dati Storici sull'Inflazione:} Dati di serie temporali che
        rappresentano i tassi di inflazione storici.
    \end{itemize}

    Il modulo di input dei dati garantisce che tutte le informazioni necessarie
    siano disponibili in un formato coerente e facilmente accessibile per le
    fasi di simulazione successive.

    \subsection{Modello di Rendimento di
    Mercato}\label{subsec:modello-di-rendimento-di-mercato}

Questo modulo è responsabile della generazione di una serie di rendimenti di
mercato per ciascuna classe di attività nel portafoglio per ogni anno della
simulazione.
Una modellazione accurata dei rendimenti di mercato è fondamentale
per simulare la crescita del portafoglio in varie condizioni di mercato.

    Un approccio comune è utilizzare i dati storici di mercato per generare una
    distribuzione di potenziali rendimenti.
    Questo può essere fatto usando varie
    tecniche:
    \begin{itemize}
        \item \textbf{Campionamento Storico (Bootstrapping):} Campionamento
        casuale dei rendimenti direttamente dai dati storici.
        Questo approccio
        presuppone che il comportamento futuro del mercato assomigli al
        comportamento passato.
        Una variante è il Block Bootstrap dove vengono
        campionati blocchi di periodi di tempo invece di singoli periodi di
        tempo.
        \item \textbf{Modellazione Parametrica:} Adattare una distribuzione
        statistica (ad esempio, distribuzione normale, distribuzione lognormale)
        ai dati storici e generare rendimenti casuali da tale distribuzione.
        Ciò
        richiede la stima dei parametri della distribuzione (ad esempio, media,
        deviazione standard).
        \item \textbf{Modelli di Serie Temporali:} Utilizzare modelli di serie
        temporali più sofisticati (ad esempio, ARMA, GARCH) per prevedere i
        rendimenti futuri sulla base dei dati passati e delle relazioni
        statistiche.
    \end{itemize}

    L'output chiave del modello di rendimento di mercato è una matrice di
    rendimenti, in cui ogni riga rappresenta un anno nella simulazione e ogni
    colonna rappresenta una diversa esecuzione della simulazione.

    Matematicamente, se \(R_{t,s}\) è il rendimento dell'asset durante l'anno
    \(t\) nella simulazione \(s\), allora il valore di un portafoglio investito
    in quell'asset crescerà di
    \[
    V_{t+1,s} = V_{t,s} \cdot (1 + R_{t,s}),
    \]
    dove \(V_{t,s}\) è il valore del portafoglio all'anno \(t\) nella
    simulazione \(s\).
    Per un portafoglio multi-asset, viene calcolata una media
    ponderata dei rendimenti di mercato specifici dell'asset.

    \subsection{Modello di Inflazione}\label{subsec:modello-di-inflazione}

L'inflazione erode il potere d'acquisto del denaro nel tempo, quindi è
essenziale incorporare l'inflazione nella simulazione.\ Il modello di inflazione
genera una serie di tassi di inflazione per ogni anno della simulazione.

Simile ai rendimenti di mercato, è possibile utilizzare vari approcci:
\begin{itemize}
    \item \textbf{Campionamento Storico:} Campionare casualmente i tassi di
    inflazione direttamente dai dati storici sull'inflazione.
    \item \textbf{Modellazione Parametrica:} Adattare una distribuzione
    statistica ai dati storici sull'inflazione e generare tassi di inflazione
    casuali da tale distribuzione.
    Una distribuzione log-normale viene spesso
    utilizzata perché i tassi di inflazione non possono essere negativi.
    \item \textbf{Tasso di Inflazione Fisso:} Assumere un tasso di inflazione
    medio costante per tutti gli anni.
    Questo fornisce uno scenario di base
    semplice.
    \item \textbf{Dati Storici Scalati:} Scala i tassi di inflazione campionati
    per soddisfare un tasso di inflazione medio specificato.
    \item \textbf{Modello Lognormale:} Il modello lognormale assume che i tassi
    di inflazione seguano una distribuzione lognormale, che è particolarmente
    adatta poiché:
    \begin{itemize}
        \item Genera solo valori positivi, coerentemente con la natura
        dell'inflazione
        \item È asimmetrica verso destra, catturando eventi di alta inflazione
        occasionale
        \item Ha una base teorica nel teorema del limite centrale per processi
        moltiplicativi
    \end{itemize}
\end{itemize}

Matematicamente, sia \(I_{t,s}\) a rappresentare il tasso di inflazione per
l'anno \(t\) nella simulazione \(s\).
Un tasso di prelievo nell'anno \(t\) di
\(W_{t,s}\) avrà il potere d'acquisto di
\[
\frac{W_{t,s}}{1+I_{t,s}}
\]
nel valore della valuta dell'anno \(t+1\).

Nel caso del modello lognormale, se \(X\) segue una distribuzione lognormale con
parametri \(\mu\) e \(\sigma\), allora:
\begin{gather*}
    E[X] = e^{\mu + \frac{\sigma^2}{2}}\\
    Var(X) = (e^{\sigma^2} - 1)e^{2\mu + \sigma^2}\\
\end{gather*}

I parametri \(\mu\) e \(\sigma\) vengono calibrati per ottenere il tasso medio
di inflazione desiderato e la varianza storica osservata.

    \subsection{Modello di Prelievo}\label{subsec:modello-di-prelievo}

Il modello di prelievo definisce la quantità di denaro prelevata dal portafoglio
ogni anno per coprire le spese di soggiorno.
La strategia di prelievo può
influenzare significativamente il tasso di successo del piano FIRE. Le strategie
di prelievo comuni includono:
    \begin{itemize}
        \item \textbf{Importo Fisso:} Un importo fisso in dollari viene
        prelevato ogni anno, potenzialmente adeguato per l'inflazione.
        \item \textbf{Strategie di Prelievo Dinamiche:} Strategie più
        sofisticate che adeguano i prelievi in base alle condizioni di mercato,
        al valore del portafoglio e ad altri fattori.
        Ad esempio, ridurre i
        prelievi durante le fasi di ribasso del mercato.
    \end{itemize}

    La rappresentazione matematica del modello di prelievo dipende dalla
    strategia scelta.
    Per un importo fisso adeguato per l'inflazione, l'importo
    del prelievo \(W_{t,s}\) per l'anno \(t\) nella simulazione \(s\) può essere
    espresso come:
    \[
    W_{t,s} = W_0 \cdot \prod_{i=1}^{t} (1 + I_{i,s}),
    \]
    dove \(W_0\) è l'importo del prelievo iniziale e \(I_{i,s}\) è il tasso di
    inflazione per l'anno \(i\) nella simulazione \(s\).
    Per un prelievo basato
    sulla percentuale, l'importo del prelievo è:
    \[
    W_{t,s} = p \cdot V_{t,s},
    \]
    dove \(p\) è la percentuale di prelievo e \(V_{t,s}\) è il valore del
    portafoglio all'inizio dell'anno \(t\) nella simulazione \(s\).

    \subsection{Modello di Evoluzione del
    Portafoglio}\label{subsec:modello-di-evoluzione-del-portafoglio}

Questo è il motore di simulazione principale che tiene traccia della performance
del portafoglio nel tempo.
Per ogni anno in ogni simulazione, l'algoritmo esegue
i seguenti passaggi:
    \begin{enumerate}
        \item \textbf{Calcola i Rendimenti degli Investimenti:} Applica i
        rendimenti di mercato per l'anno alle attività del portafoglio,
        adeguando per l'asset allocation.
        \item \textbf{Adatta per l'Inflazione:} Prendi in considerazione
        l'impatto dell'inflazione dell'anno in corso.
        \item \textbf{Preleva Fondi:} Preleva l'importo pianificato in base alla
        strategia di prelievo.
        \item \textbf{Calcola Imposte e Spese:} Calcola e detrai eventuali
        imposte applicabili (ad esempio, l'imposta sulle plusvalenze) e le spese
        (ad esempio, le commissioni di gestione).
        \item \textbf{Aggiorna il Valore del Portafoglio:} Aggiorna il valore
        del portafoglio per riflettere i rendimenti degli investimenti, i
        prelievi, le imposte e le spese.
        \item \textbf{Aggiorna le Riserve di Liquidità:} Aggiorna le riserve di
        liquidità se vengono trasferiti fondi da/verso il portafoglio di
        investimenti.
    \end{enumerate}

    L'equazione fondamentale che governa l'evoluzione del portafoglio può essere
    espressa come:
    \[
    V_{t+1,s} = V_{t,s} \cdot (1 + R_{t,s}) - W_{t,s} - T_{t,s},
    \]
    dove:
    \begin{itemize}
        \item \(V_{t+1,s}\) è il valore del portafoglio alla fine dell'anno
        \(t\) nella simulazione \(s\).
        \item \(V_{t,s}\) è il valore del portafoglio all'inizio dell'anno \(t\)
        nella simulazione \(s\).
        \item \(R_{t,s}\) è il rendimento di mercato per l'anno \(t\) nella
        simulazione \(s\).
        \item \(W_{t,s}\) è l'importo del prelievo per l'anno \(t\) nella
        simulazione \(s\).
        \item \(T_{t,s}\) sono le imposte e le spese totali per l'anno \(t\)
        nella simulazione \(s\).
    \end{itemize}

    L'algoritmo ripete questi passaggi per ogni anno nell'orizzonte di
    simulazione e per ogni esecuzione di simulazione.

    \subsection{Calcolo del Tasso di
    Successo}\label{subsec:calcolo-del-tasso-di-successo}

Dopo aver completato tutte le esecuzioni della simulazione, il tasso di successo
viene calcolato determinando la percentuale di simulazioni in cui il portafoglio
rimane solvibile alla fine del periodo di simulazione.

    Una simulazione è tipicamente considerata riuscita se il valore del
    portafoglio rimane al di sopra dello zero (o di una soglia minima
    predefinita) alla fine dell'orizzonte di simulazione.
    Il tasso di successo
    viene quindi calcolato come:
    \[
    \text{Tasso di Successo} = \frac{\text{Numero di Simulazioni Riuscite}}{\text{Numero Totale di Simulazioni}} \times 100\%
    \]
    Il tasso di successo fornisce una misura probabilistica della fattibilità
    del piano FIRE. Un tasso di successo più elevato indica una maggiore
    probabilità di raggiungere l'indipendenza finanziaria e di mantenere lo
    stile di vita desiderato per tutto il periodo di pensionamento.