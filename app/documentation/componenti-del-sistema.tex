\subsection{SyncWorker}\label{subsec:syncworker} Gestisce la sincronizzazione.
Usa Hilt.

\begin{lstlisting}[caption=SyncWorker.kt, label=lst:syncworker, basicstyle=\ttfamily\scriptsize, breaklines=true, breakatwhitespace=true, tabsize=4]
@HiltWorker class SyncWorker<T> @AssistedInject constructor( @Assisted context:
Context, @Assisted workerParams: WorkerParameters, private val
syncQueueItemRepository: SyncQueueItemRepository, private val
firestoreRepository: FirestoreRepository, private val localRepository:
LocalDatabaseRepository<T> ) : CoroutineWorker(context, workerParams) {
companion object { private const val TAG = "SyncWorker" }

    override suspend fun doWork(): Result = coroutineScope { try { Log.d(TAG,
        "Starting sync work") cleanupStuckSyncingItems()

            val currentTime = System.currentTimeMillis() val pendingItems =
            syncQueueItemRepository.getPendingItems(currentTime) Log.d(TAG,
            "Found ${pendingItems.size} pending items ready for processing")

            if (pendingItems.isEmpty()) { Log.d(TAG, "No pending items,
                completing successfully") return@coroutineScope Result.success()
                }

            val results = processBatches(pendingItems) Log.d(TAG, "Processed
            ${results.size} items")

            val errorCount = results.count { it is SyncOperationResult.Error }
            val successCount = results.count { it is SyncOperationResult.Success
            } val retryCount = results.count { it is SyncOperationResult.Retry }
            val staleCount = results.count { it is SyncOperationResult.StaleData
            }

            Log.d(TAG, "Sync results - Success: $successCount, Errors:
            $errorCount, Retries: $retryCount, Stale: $staleCount")

            handleFailedItems()

            return@coroutineScope when { errorCount > 0 -> { Log.w(TAG, "Some
                operations failed, scheduling retry") Result.retry() }
                staleCount > 0 -> { Log.d(TAG, "Some items were stale and have
                been handled") Result.success() } else -> { Log.d(TAG, "All
                operations completed successfully") Result.success() } } } catch
                (e: Exception) { handleWorkerError(e) } } // ... (rest of the
                code) }
\end{lstlisting}

\textbf{doWork()}: Sovrascrive \texttt{doWork()}.
Passaggi:
\begin{enumerate}
    \item \textbf{cleanupStuckSyncingItems()}: Pulisce elementi bloccati in
    \texttt{SYNCING}.
    \item \textbf{Recupera elementi in sospeso}: Ottiene elementi pronti.
    \item \textbf{Elabora i batch}: Divide in batch.
    \item \textbf{Gestisce i risultati}: Gestisce successo, errore, retry.
    \item \textbf{handleFailedItems()}: Elimina tentativi massimi superati.
    \item \textbf{Restituisce il risultato}: \texttt{Result.success()},
    \texttt{Result.retry()} o \texttt{Result.failure()}.
\end{enumerate}

\textbf{Operazioni di sincronizzazione}:

\begin{lstlisting}[caption=Operazioni di Sincronizzazione, label=lst:sync_operations, basicstyle=\ttfamily\scriptsize, breaklines=true, breakatwhitespace=true, tabsize=4]
    private suspend fun performAddOperation(item: SyncQueueItem):
        SyncOperationResult { Log.d(TAG, "Performing ADD operation for item
        ${item.id}") return firestoreRepository.addDocument(collectionPath =
        item.collection, data = item.payload, documentId =
        item.id).first().fold( onSuccess = { Log.d(TAG, "ADD operation
        successful for item ${item.id}"); SyncOperationResult.Success(item.id)
        }, onFailure = { error -> Log.e(TAG, "ADD operation failed for item
        ${item.id}", error); throw error } ) }

    private suspend fun performUpdateOperation(item: SyncQueueItem):
        SyncOperationResult { Log.d(TAG, "Performing UPDATE operation for item
        ${item.id}") try { val currentDocResult =
        firestoreRepository.getDocumentById(collectionPath = item.collection,
        documentId = item.id).first() val currentDoc =
        currentDocResult.getOrNull() ?: throw IllegalStateException("Remote
        document not found or null") val remoteTimestamp =
        (currentDoc["updatedAt"] as? Double)?.toLong() ?: throw
        IllegalStateException("Remote document missing updatedAt timestamp") val
        localTimestamp = item.createdAt if (remoteTimestamp > localTimestamp) {
        Log.d("UpdateUserSettingsUseCase", "Remote document is newer (remote:
        $remoteTimestamp, local: $localTimestamp)") return
        SyncOperationResult.StaleData(item.id) } return
        firestoreRepository.updateDocument(collectionPath = item.collection,
        data = item.payload, documentId = item.id).first().fold( onSuccess = {
        Log.d(TAG, "UPDATE operation successful for item ${item.id}");
        SyncOperationResult.Success(item.id) }, onFailure = { error ->
        Log.e(TAG, "UPDATE operation failed for item ${item.id}", error); throw
        error } ) } catch (e: Exception) { Log.e(TAG, "Error during UPDATE
        operation for item ${item.id}", e); throw e } }

    private suspend fun performDeleteOperation(item: SyncQueueItem):
        SyncOperationResult { Log.d(TAG, "Performing DELETE operation for item
        ${item.id}") return firestoreRepository.deleteDocument(collectionPath =
        item.collection, documentId = item.id).first().fold( onSuccess = {
        Log.d(TAG, "DELETE operation successful for item ${item.id}");
        SyncOperationResult.Success(item.id) }, onFailure = { error ->
        Log.e(TAG, "DELETE operation failed for item ${item.id}", error); throw
        error } ) }
\end{lstlisting}

\subsection{SyncQueueItem}\label{subsec:syncqueueitem} Rappresenta un'operazione
di sincronizzazione.

\begin{lstlisting}[caption=SyncQueueItem.kt, label=lst:syncqueueitem, basicstyle=\ttfamily\scriptsize, breaklines=true, breakatwhitespace=true, tabsize=4]
package com.unimib.ignitionfinance.data.local.entity

import androidx.room.ColumnInfo import androidx.room.Entity import
androidx.room.PrimaryKey import
com.unimib.ignitionfinance.data.local.utils.SyncStatus

@Entity(tableName = "sync_queue_items") data class SyncQueueItem ( @PrimaryKey
val id: String, @ColumnInfo(name = "collection") val collection: String,
@ColumnInfo(name = "payload") val payload: Map<String, Any>, @ColumnInfo(name =
"operation") val operationType: String, @ColumnInfo(name = "status") val status:
SyncStatus = SyncStatus.PENDING, @ColumnInfo(name = "attempts") val attempts:
Int = 0, @ColumnInfo(name = "created_at") val createdAt: Long =
System.currentTimeMillis(), @ColumnInfo(name = "scheduled_for") val
scheduledFor: Long = System.currentTimeMillis() )
\end{lstlisting}

\textbf{Campi principali}:
\begin{itemize}
    \item \texttt{id}: ID univoco.
    \item \texttt{collection}: Nome collezione Firestore.
    \item \texttt{payload}: Dati da sincronizzare.
    \item \texttt{operationType}: ADD, UPDATE, DELETE\@.
    \item \texttt{status}: PENDING, SYNCING, SUCCEEDED, FAILED\@.
    \item \texttt{attempts}: Numero di tentativi.
    \item \texttt{createdAt}: Timestamp creazione.
    \item \texttt{scheduledFor}: Timestamp schedulato.
\end{itemize}

\subsection{SyncQueueItemRepository}\label{subsec:syncqueueitemrepository}
Fornisce accesso al DB locale.

\begin{lstlisting}[caption=SyncQueueItemDao.kt, label=lst:syncqueueitemdao, basicstyle=\ttfamily\scriptsize, breaklines=true, breakatwhitespace=true, tabsize=4]
package com.unimib.ignitionfinance.data.local.database

import androidx.room.Dao import androidx.room.Delete import androidx.room.Insert
import androidx.room.OnConflictStrategy import androidx.room.Query import
androidx.room.Update import
com.unimib.ignitionfinance.data.local.entity.SyncQueueItem import
com.unimib.ignitionfinance.data.local.utils.SyncStatus

@Dao interface SyncQueueItemDao { @Insert(onConflict =
OnConflictStrategy.REPLACE) suspend fun insert(item: SyncQueueItem)

    @Update suspend fun update(item: SyncQueueItem)

    @Delete suspend fun delete(item: SyncQueueItem)

    @Query("SELECT * FROM sync_queue_items WHERE id = :id") suspend fun
    getById(id: String): SyncQueueItem?

    @Query("SELECT * FROM sync_queue_items WHERE status = :status") suspend fun
    getByStatus(status: SyncStatus): List<SyncQueueItem>

    @Query("SELECT * FROM sync_queue_items WHERE status = :status AND collection
    = :collection") suspend fun getByStatusAndCollection(status: SyncStatus,
    collection: String): List<SyncQueueItem>

    @Query("SELECT * FROM sync_queue_items WHERE scheduled_for <= :timestamp AND
    status = :status") suspend fun getPendingItems(timestamp: Long =
    System.currentTimeMillis(), status: SyncStatus = SyncStatus.PENDING):
    List<SyncQueueItem>

    @Query("SELECT * FROM sync_queue_items WHERE attempts >= :maxAttempts AND
    status = :status") suspend fun getFailedItems(maxAttempts: Int, status:
    SyncStatus = SyncStatus.FAILED): List<SyncQueueItem> }
\end{lstlisting}

\textbf{Metodi principali}:
\begin{itemize}
    \item \texttt{insert(item)}: Inserisce.
    \item \texttt{update(item)}: Aggiorna.
    \item \texttt{delete(item)}: Elimina.
    \item \texttt{getById(id)}: Recupera per ID\@.
    \item \texttt{getByStatus(status)}: Recupera per stato.
    \item \texttt{getPendingItems(timestamp)}: Recupera in sospeso.
\end{itemize}

\subsection{SyncOperationScheduler}\label{subsec:syncoperationscheduler}
Gestisce pianificazione con WorkManager.

\begin{lstlisting}[caption=SyncOperationScheduler.kt, label=lst:syncoperationscheduler, basicstyle=\ttfamily\scriptsize, breaklines=true, breakatwhitespace=true, tabsize=4]
package com.unimib.ignitionfinance.data.remote.worker

import android.content.Context import androidx.work.BackoffPolicy import
androidx.work.Constraints import androidx.work.ExistingPeriodicWorkPolicy import
androidx.work.NetworkType import androidx.work.OneTimeWorkRequestBuilder import
androidx.work.PeriodicWorkRequestBuilder import androidx.work.WorkManager import
java.util.concurrent.TimeUnit

object SyncOperationScheduler { const val TAG = "SyncOperationScheduler" const
    val SYNC_WORK_NAME = "sync_operation_work" const val MAX_RETRIES = 3 const
    val SYNC_TIMEOUT_MS = 30000L const val BATCH_SIZE = 10 const val
    BATCH_DELAY_MS = 1000L const val INITIAL_BACKOFF_DELAY_MS = 10000L const val
    PERIODIC_SYNC_INTERVAL = 15L const val MIN_BACKOFF_DELAY_MS = 5000L const
    val MAX_BACKOFF_DELAY_MS = 300000L

    inline fun <reified T> schedule(context: Context, constraints: Constraints =
        getDefaultConstraints()) { Log.d(TAG, "Scheduling periodic sync work
        with interval: $PERIODIC_SYNC_INTERVAL minutes") try { val
        syncWorkRequest =
        PeriodicWorkRequestBuilder<SyncWorker<T>>(repeatInterval =
        PERIODIC_SYNC_INTERVAL, repeatIntervalTimeUnit = TimeUnit.MINUTES,
        flexTimeInterval = 5, flexTimeIntervalUnit = TimeUnit.MINUTES)
        .setConstraints(constraints)
        .setBackoffCriteria(BackoffPolicy.EXPONENTIAL, INITIAL_BACKOFF_DELAY_MS,
        TimeUnit.MILLISECONDS) .build() Log.d(TAG, "Work request built with
        constraints: ${constraints.requiredNetworkType}")
        WorkManager.getInstance(context).enqueueUniquePeriodicWork(SYNC_WORK_NAME,
        ExistingPeriodicWorkPolicy.REPLACE, syncWorkRequest) Log.i(TAG,
        "Periodic sync work scheduled successfully") } catch (e: Exception) {
        Log.e(TAG, "Error scheduling periodic sync work", e) } }

    inline fun <reified T> scheduleOneTime(context: Context, constraints:
        Constraints = getDefaultConstraints()) { Log.d(TAG, "Scheduling one-time
        sync work with initial delay: $initialDelay ms") try { val
        syncWorkRequest = OneTimeWorkRequestBuilder<SyncWorker<T>>()
        .setConstraints(constraints)
        .setBackoffCriteria(BackoffPolicy.EXPONENTIAL,
        INITIAL_BACKOFF_DELAY_MS.coerceIn(MIN_BACKOFF_DELAY_MS,
        MAX_BACKOFF_DELAY_MS), TimeUnit.MILLISECONDS)
        .setInitialDelay(initialDelay, TimeUnit.MILLISECONDS) .build()
        Log.d(TAG, "One-time work request built with constraints:
        ${constraints.requiredNetworkType}")
        WorkManager.getInstance(context).enqueue(syncWorkRequest) Log.i(TAG,
        "One-time sync work scheduled successfully") } catch (e: Exception) {
        Log.e(TAG, "Error scheduling one-time sync work", e) } }

    fun cancel(context: Context) { Log.d(TAG, "Cancelling sync work:
        $SYNC_WORK_NAME") try {
        WorkManager.getInstance(context).cancelUniqueWork(SYNC_WORK_NAME)
        Log.i(TAG, "Sync work cancelled successfully") } catch (e: Exception) {
        Log.e(TAG, "Error cancelling sync work", e) } }

    fun getDefaultConstraints() =
Constraints.Builder().setRequiredNetworkType(NetworkType.CONNECTED).setRequiresBatteryNotLow(true).build()
}
\end{lstlisting}

\textbf{Metodi principali}:
\begin{itemize}
    \item \texttt{schedule(context, constraints)}: Pianifica periodica.
    \item \texttt{scheduleOneTime(context, constraints)}: Pianifica una tantum.
    \item \texttt{cancel(context)}: Annulla.
    \item \texttt{getDefaultConstraints()}: Restituisce constraints predefinite.
\end{itemize}

\subsection{SyncOperationResult}\label{subsec:syncoperationresult} Risultato di
un'operazione.

\begin{lstlisting}[caption=SyncOperationResult.kt, label=lst:syncoperationresult, basicstyle=\ttfamily\scriptsize, breaklines=true, breakatwhitespace=true, tabsize=4]
package com.unimib.ignitionfinance.data.remote.worker

sealed class SyncOperationResult { data class Success(val itemId: String) :
    SyncOperationResult() data class Retry(val itemId: String, val error:
    Throwable) : SyncOperationResult() data class Error(val itemId: String, val
    error: Throwable) : SyncOperationResult() data class StaleData(val itemId:
    String) : SyncOperationResult() }
\end{lstlisting}

\textbf{Possibili risultati}:
\begin{itemize}
    \item \texttt{Success(itemId)}: Successo.
    \item \texttt{Retry(itemId, error)}: Riprova.
    \item \texttt{Error(itemId, error)}: Errore.
    \item \texttt{StaleData(itemId)}: Dati obsoleti.
\end{itemize}